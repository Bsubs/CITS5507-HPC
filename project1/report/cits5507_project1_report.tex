% Options for packages loaded elsewhere
\PassOptionsToPackage{unicode}{hyperref}
\PassOptionsToPackage{hyphens}{url}
%
\documentclass[
]{article}
\usepackage{amsmath,amssymb}
\usepackage{iftex}
\ifPDFTeX
  \usepackage[T1]{fontenc}
  \usepackage[utf8]{inputenc}
  \usepackage{textcomp} % provide euro and other symbols
\else % if luatex or xetex
  \usepackage{unicode-math} % this also loads fontspec
  \defaultfontfeatures{Scale=MatchLowercase}
  \defaultfontfeatures[\rmfamily]{Ligatures=TeX,Scale=1}
\fi
\usepackage{lmodern}
\ifPDFTeX\else
  % xetex/luatex font selection
\fi
% Use upquote if available, for straight quotes in verbatim environments
\IfFileExists{upquote.sty}{\usepackage{upquote}}{}
\IfFileExists{microtype.sty}{% use microtype if available
  \usepackage[]{microtype}
  \UseMicrotypeSet[protrusion]{basicmath} % disable protrusion for tt fonts
}{}
\makeatletter
\@ifundefined{KOMAClassName}{% if non-KOMA class
  \IfFileExists{parskip.sty}{%
    \usepackage{parskip}
  }{% else
    \setlength{\parindent}{0pt}
    \setlength{\parskip}{6pt plus 2pt minus 1pt}}
}{% if KOMA class
  \KOMAoptions{parskip=half}}
\makeatother
\usepackage{xcolor}
\usepackage[margin=1in]{geometry}
\usepackage{color}
\usepackage{fancyvrb}
\newcommand{\VerbBar}{|}
\newcommand{\VERB}{\Verb[commandchars=\\\{\}]}
\DefineVerbatimEnvironment{Highlighting}{Verbatim}{commandchars=\\\{\}}
% Add ',fontsize=\small' for more characters per line
\usepackage{framed}
\definecolor{shadecolor}{RGB}{248,248,248}
\newenvironment{Shaded}{\begin{snugshade}}{\end{snugshade}}
\newcommand{\AlertTok}[1]{\textcolor[rgb]{0.94,0.16,0.16}{#1}}
\newcommand{\AnnotationTok}[1]{\textcolor[rgb]{0.56,0.35,0.01}{\textbf{\textit{#1}}}}
\newcommand{\AttributeTok}[1]{\textcolor[rgb]{0.13,0.29,0.53}{#1}}
\newcommand{\BaseNTok}[1]{\textcolor[rgb]{0.00,0.00,0.81}{#1}}
\newcommand{\BuiltInTok}[1]{#1}
\newcommand{\CharTok}[1]{\textcolor[rgb]{0.31,0.60,0.02}{#1}}
\newcommand{\CommentTok}[1]{\textcolor[rgb]{0.56,0.35,0.01}{\textit{#1}}}
\newcommand{\CommentVarTok}[1]{\textcolor[rgb]{0.56,0.35,0.01}{\textbf{\textit{#1}}}}
\newcommand{\ConstantTok}[1]{\textcolor[rgb]{0.56,0.35,0.01}{#1}}
\newcommand{\ControlFlowTok}[1]{\textcolor[rgb]{0.13,0.29,0.53}{\textbf{#1}}}
\newcommand{\DataTypeTok}[1]{\textcolor[rgb]{0.13,0.29,0.53}{#1}}
\newcommand{\DecValTok}[1]{\textcolor[rgb]{0.00,0.00,0.81}{#1}}
\newcommand{\DocumentationTok}[1]{\textcolor[rgb]{0.56,0.35,0.01}{\textbf{\textit{#1}}}}
\newcommand{\ErrorTok}[1]{\textcolor[rgb]{0.64,0.00,0.00}{\textbf{#1}}}
\newcommand{\ExtensionTok}[1]{#1}
\newcommand{\FloatTok}[1]{\textcolor[rgb]{0.00,0.00,0.81}{#1}}
\newcommand{\FunctionTok}[1]{\textcolor[rgb]{0.13,0.29,0.53}{\textbf{#1}}}
\newcommand{\ImportTok}[1]{#1}
\newcommand{\InformationTok}[1]{\textcolor[rgb]{0.56,0.35,0.01}{\textbf{\textit{#1}}}}
\newcommand{\KeywordTok}[1]{\textcolor[rgb]{0.13,0.29,0.53}{\textbf{#1}}}
\newcommand{\NormalTok}[1]{#1}
\newcommand{\OperatorTok}[1]{\textcolor[rgb]{0.81,0.36,0.00}{\textbf{#1}}}
\newcommand{\OtherTok}[1]{\textcolor[rgb]{0.56,0.35,0.01}{#1}}
\newcommand{\PreprocessorTok}[1]{\textcolor[rgb]{0.56,0.35,0.01}{\textit{#1}}}
\newcommand{\RegionMarkerTok}[1]{#1}
\newcommand{\SpecialCharTok}[1]{\textcolor[rgb]{0.81,0.36,0.00}{\textbf{#1}}}
\newcommand{\SpecialStringTok}[1]{\textcolor[rgb]{0.31,0.60,0.02}{#1}}
\newcommand{\StringTok}[1]{\textcolor[rgb]{0.31,0.60,0.02}{#1}}
\newcommand{\VariableTok}[1]{\textcolor[rgb]{0.00,0.00,0.00}{#1}}
\newcommand{\VerbatimStringTok}[1]{\textcolor[rgb]{0.31,0.60,0.02}{#1}}
\newcommand{\WarningTok}[1]{\textcolor[rgb]{0.56,0.35,0.01}{\textbf{\textit{#1}}}}
\usepackage{graphicx}
\makeatletter
\def\maxwidth{\ifdim\Gin@nat@width>\linewidth\linewidth\else\Gin@nat@width\fi}
\def\maxheight{\ifdim\Gin@nat@height>\textheight\textheight\else\Gin@nat@height\fi}
\makeatother
% Scale images if necessary, so that they will not overflow the page
% margins by default, and it is still possible to overwrite the defaults
% using explicit options in \includegraphics[width, height, ...]{}
\setkeys{Gin}{width=\maxwidth,height=\maxheight,keepaspectratio}
% Set default figure placement to htbp
\makeatletter
\def\fps@figure{htbp}
\makeatother
\setlength{\emergencystretch}{3em} % prevent overfull lines
\providecommand{\tightlist}{%
  \setlength{\itemsep}{0pt}\setlength{\parskip}{0pt}}
\setcounter{secnumdepth}{-\maxdimen} % remove section numbering
\ifLuaTeX
  \usepackage{selnolig}  % disable illegal ligatures
\fi
\IfFileExists{bookmark.sty}{\usepackage{bookmark}}{\usepackage{hyperref}}
\IfFileExists{xurl.sty}{\usepackage{xurl}}{} % add URL line breaks if available
\urlstyle{same}
\hypersetup{
  pdftitle={CITS5507 Project 1 Report},
  pdfauthor={Joo Kai Tay (22489437)},
  hidelinks,
  pdfcreator={LaTeX via pandoc}}

\title{CITS5507 Project 1 Report}
\author{Joo Kai Tay (22489437)}
\date{2023-09-18}

\begin{document}
\maketitle

\begin{Shaded}
\begin{Highlighting}[]
\FunctionTok{library}\NormalTok{(ggplot2)}
\FunctionTok{library}\NormalTok{(readxl)}
\end{Highlighting}
\end{Shaded}

\begin{Shaded}
\begin{Highlighting}[]
\NormalTok{df }\OtherTok{\textless{}{-}} \FunctionTok{read\_excel}\NormalTok{(}\StringTok{"data.xlsx"}\NormalTok{)}
\end{Highlighting}
\end{Shaded}

\hypertarget{project-1-report}{%
\section{Project 1 Report}\label{project-1-report}}

\hypertarget{project-overview}{%
\subsection{Project Overview}\label{project-overview}}

This project aims to test various sequential and parallel
implementations of the Fish School Behaviour (FSB) algorithm. The
performance for each experiment will be measured by the time taken to
execute the code.

\hypertarget{fish-simulation}{%
\subsection{Fish simulation}\label{fish-simulation}}

The file fish.h contains the definition of a fish in this simulation.
The struct Fish contains the following parameters: 1. double euclDist:
This is the euclidean distance of the fish from the origin in the
previous step 2. double x\_c: This holds the x coordinate of the fish in
the current step 3. double y\_c: This holds the y coordinate of the fish
in the current step 4. double weight\_c: This holds the weight of the
fish in the current step 5. double weight\_p: This holds the weight of
the fish in the previous step

\begin{verbatim}
typedef struct fish {
    double euclDist;
    double x_c;
    double y_c;
    double weight_c;
    double weight_p;
} Fish;
\end{verbatim}

fish.h also contains declarations for functions that are implemented in
fish.c

This function dynamically allocates memory in the heap for an array of
Fish structures and initializes each fish with random coordinates within
a 200 x 200 square. The parameter numfish allows the user to specify the
size of the array.

\begin{verbatim}
Fish* initializeFish(int numfish)
\end{verbatim}

This function simulates a fish eating in the current step. The weight of
the fish is bounded between 0 and 2w.

\begin{verbatim}
void eat(Fish* fish1, double maxObj, int step)
\end{verbatim}

This function simulates a fish swimming in the current step.

\begin{verbatim}
void swim(Fish* fish1)
\end{verbatim}

\hypertarget{experiments}{%
\subsection{Experiments}\label{experiments}}

The experiments make use of functions declared in \texttt{sequential.h}
and \texttt{parallel\_functions.h}. Each experiment has its
corresponding \texttt{experimentX.c} and \texttt{experimentX.sh} files.
The experiments are run on setonix using the command:
\texttt{sbatch\ experimentX.sh}

Each experiment was ran 10 times and the results show are an average of
those 10 simulations.

\hypertarget{experiment-1---time-taken-as-number-of-threads-increase}{%
\subsubsection{Experiment 1 - Time taken as number of threads
increase}\label{experiment-1---time-taken-as-number-of-threads-increase}}

Experiment 1 investigates the effect of changing the number of threads
on the time take for computation. Number of threads: 1, 2, 4, 8, 16, 32
Number of steps: 10,000 Number of fish: 10,000

\includegraphics{cits5507_project1_report_files/figure-latex/unnamed-chunk-3-1.pdf}

\hypertarget{experiment-2---investigating-time-vs-number-of-fish}{%
\subsubsection{Experiment 2 - Investigating time vs number of
fish}\label{experiment-2---investigating-time-vs-number-of-fish}}

Experiment 2 investigates the increase in time taken as the number of
fish increases. Number of threads: 16 Number of steps: 10,000 Number of
fish: 5000, 10000, 15000, 20000, 25000

The plot below shows the increase in time as the number of fish
increases is linear.

\includegraphics{cits5507_project1_report_files/figure-latex/unnamed-chunk-4-1.pdf}
\#\#\# Experiment 3 - Comparing Steps and Fishes Experiment 3
investigates the increase in time taken as the number of steps
increases. Number of threads: 16 Number of steps: 5000, 10000, 15000,
20000, 25000 Number of fish: 10,000

The plot below shows that the increase in time as the number of steps
increases is also linear.
\includegraphics{cits5507_project1_report_files/figure-latex/unnamed-chunk-5-1.pdf}

\includegraphics{cits5507_project1_report_files/figure-latex/unnamed-chunk-6-1.pdf}

\hypertarget{experiment-4---thread-scheduling-methods}{%
\subsubsection{Experiment 4 - Thread Scheduling
Methods}\label{experiment-4---thread-scheduling-methods}}

\includegraphics{cits5507_project1_report_files/figure-latex/unnamed-chunk-7-1.pdf}

\hypertarget{experiment-5---tasks-and-sections}{%
\subsubsection{Experiment 5 - Tasks and
Sections}\label{experiment-5---tasks-and-sections}}

\includegraphics{cits5507_project1_report_files/figure-latex/unnamed-chunk-8-1.pdf}

\end{document}
